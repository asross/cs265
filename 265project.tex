   
\documentclass[11pt]{article}
\renewcommand{\baselinestretch}{1.05}
\usepackage{amsmath,amsthm,verbatim,amssymb,amsfonts,amscd, graphicx}
\usepackage{graphics}
\topmargin0.0cm
\headheight0.0cm
\headsep0.0cm
\oddsidemargin0.0cm
\textheight23.0cm
\textwidth16.5cm
\footskip1.0cm
\theoremstyle{plain}
\newtheorem{theorem}{Theorem}
\newtheorem{corollary}{Corollary}
\newtheorem{lemma}{Lemma}
\newtheorem{proposition}{Proposition}
\newtheorem*{surfacecor}{Corollary 1}
\newtheorem{conjecture}{Conjecture} 
\newtheorem{question}{Question} 
\theoremstyle{definition}
\newtheorem{definition}{Definition}

 \begin{document}
 


\title{Template}
\author{Mali Akmanalp, Sophie Hilgard, Andrew Ross}
\maketitle

\section{Base Case}

Assume
$n$ items in total DB \\
$NC$ items that fit in cache\\
$NM$ items that fit in memtable\\
$R$ ratio between layers of LSM tree such that \\
$L1 = R * NM$ \\
$L2 = R^2 * NM \dots$ \\ 

We can solve for $j$ the total number of layers required to store all the data: \\
$$NM * \frac{1-R^j}{1-R} = n$$
$$j = \frac{log(1-n*\frac{1-R}{NM})}{log R}$$

The average cost of a write remains the same as for the basic LSM tree case:
$$
\textrm{log}_{R} \frac{n}{NM}
$$

The average cost of a read, we consider probabilistically over all possible locations of the read item, assuming a random distribution of reads: \\
Probability that read is in memtable = $\frac{NM}{n}  = p(mt)$\\
Probability that read is in cache = $\frac{NC}{n} = p(cache)$ \\
Probability that read is in L1 but not in cache = $ \frac{NM * R - \frac{NM * R}{NM * \frac{1-(R^j-1)}{1-R}} * NC}{n}  = p(L1)$\\
Where the numerator is the number of items that are in the first layer $$NM * R$$ minus the proportion of items from that layer that are probabilistically in the cache already $$\frac{NM * R}{NM * \frac{1-(R^j-1)}{1-R}} * NC$$
Where here the $R^j -1$ comes from the fact that items already in memtable (L0) are not allowed to occupy the cache.

Expected cost of read = $p(mt) * 0  + p(cache) + 0 + \sum_{i=1}^j p(Li) * i$

\section{Skewed Reads}

Now consider the case for skewed reads, where we say $d_{hf}$ ($d_{lf}$) percent of the data receives $r_{hf}$ ($r_{lf}$) percent of the reads (where $d_{hf} + d_{lf} = 1$ and $r_{hf} + r_{lf} = 1$). On average, we can assume that the cache contains $r_{hf} * NC$ items from $d_{hf} * n$ and $r_{lf} * NC$ items from $d_{lf} * n$. Then the expected cost of a read is dependent on whether the data item being read is in $d_{hf} * n$ or $d_{lf} * n$ as the probability of a cache hit varies.\\
For data in $d_{hf} * n$, \\ \\
Probability that read is in memtable = $\frac{NM*d_{hf}}{d_{hf} *n}  = p(mt)$\\
Probability that read is in cache = $\frac{r_{hf} * NC}{d_{hf} * n} = p(cache_{hf})$ \\
Probability that read is in L1 but not in cache = $ \frac{NM * R*d_{hf} - \frac{NM * R}{NM * \frac{1-(R^j-1)}{1-R}} * r_{hf} * NC}{d_{hf} * n}  = p(L1_{hf})$ \\
Expected cost of read on item in $d_{hf}$: $E[C_{hf}]= p(mt) * 0  + p(cache_{hf}) + 0 + \sum_{i=1}^j p(Li_{hf}) * i$\\ \\
Concretely, consider where we have 3 levels and 800 total items with a cache of size 10 and a ratio of 2  (for L0=100, L1 = 200, L2 = 400 items), with $d_{hf} = .2$ and $d_{lf} = .8$ and $r_{hf}$ = .8 and $r_{lf}$ = .2. Then the cache on average contains 8 items from $d_{hf} * n$ and 2 items from $d_{lf}*n$. If we execute a read on one of the 200 items in $d_{hf}$, then, there is a $\frac{8}{200}$ chance that that item is in the cache. If we execute a read on one of the $200*\frac{1}{4} = 50$ items of  $d_{hf} * n$ in L1, we expect that $\frac{2}{6} * 8$ of those items would have actually been found already in cache, as this level contains $\frac{2}{6}$ of all of the items not in the memtable. Then the probability that a read is found in L1 is the proportion of the $d_{hf} * n = 160$ items that will reside in L1 but not in the cache, which is $\frac{40 - \frac{2}{6} * 8}{160}$. \\ \\
The expected cost of a read on an item in $d_{lf}$ can be enumerated analogously, and we combine the expectation of reads in $d_{hf}$ and $d_{lf}$ as: \\
Expected cost of read = $r_{hf} * E[C_{hf}] + r_{lf} * E[C_{lf}]$ \\ \\
We can also define $r_{lf}$ in terms of $r_{hf}$ as $r_{hf} - 1$ and  $d_{lf}$ in terms of $d_{hf}$ as $d_{hf} - 1$ . (Doing this will make the effect that moving these parameters in one direction or the other has more obvious in the total overall formula.)
\section{Bloom Filters}

\section{Variable Cache Size}
 
\end{document}